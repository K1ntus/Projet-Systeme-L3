\documentclass{article}
\usepackage[utf8]{inputenc}

\title{Rapport projet système}
\author{Jordane Masson \\ Nathan Bonnaud \\ Sarah Portejoie}
\date{Décembre 2018}

\begin{document}

\maketitle

\section{Introduction}
Dans le cadre du projet de programmation système, nous avons du réaliser deux parties, A et B, dans le but de développer un framework de tests. Pour réaliser ce travail, nous avons utilisé l'outil de collaboration GitHub. Nous allons voir comment et quels choix nous avons fait pour  créer ce framework.
    
   
\section{Partie A}   
\subsection{Structure testfw\_t}
Le choix des champs d'une telle structure a été induite par les arguments de la fonction testfw\_init(). La structure est composée d'un nom de programme, un nom de fichier, une commande, un entier définissant le temps maximum d'un test avant d'être en "timeout", une variable booléenne silent permettant d'avoir un affichage ou non sur le terminal, et enfin d'une variable booléenne verbose servant à afficher plus d'info sur le fonctionnement interne du framework de test.
\subsection{Fonction testfw\_init()}
\subsection{Fonction testfw\_free()}
\subsection{Fonction testfw\_length()}
\subsection{Fonction testfw\_get()}
\subsection{Fonction testfw\_register\_func()}
\subsection{Fonction testfw\_register\_symb()}
\subsection{Fonction testfw\_register\_suite()}
 
\section{Partie B}
\subsection{Fonction testfw\_run\_all()}
    
\end{document}

